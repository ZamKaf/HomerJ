\section{Функциональность комплекса }

\begin{description}

\item[ПО «Агент»: блок сбора данных]

Собирает следующие данные с рабочей станции: Список процессов со всем необходимыми данными (имя пользователя, затраты этим процессом CPU и оперативной памяти в процентах от общего значения) Загруженность CPU в процентах от общего значения Затраты оперативной памяти в процентах от общего значения и в килобайтах

\item[ПО «Агент»: блок общения с ПО «Монитор»]

Выполняет следующие задачи:
•	Приём запросов от ПО «Монитор» на получение данных 
•	Приём запросов от «ПО Монитор» на проверку наличия соединения.
•	Передача в «ПО Монитор» сообщений о включении/выключении агента 
•	Приём ответа на сообщение о включении агента
•	Передача в «ПО Монитор» собранных данных
\item[ПО «Агент»: блок работы сервиса]

Набор функций, необходимых для функционирования этой программы, как сервиса.

\item[ПО «Агент»: рабочий блок]

Блок, обеспечивающий работу приложения, и использующий функции из остальных блоков.

\item[ПО «Монитор» (сервис): блок общения с ПО «Агент»]

Выполняет следующие задачи:
•	Передача в «ПО Агент» запросов на получение данных
•	Передача в «ПО Агент» запросов на установление соединения
•	Получение от «ПО Агент» собранных данных
•	Приём полученных от ПО «Агент» сообщений о его включении/выключении 
•	Отправка ответов на сообщение о включении
•	Обработка мультипоточности запросов и формирование из них очереди

\item[ПО «Монитор» (сервис): блок общения с графической утилитой]

Выполняет следующие задачи:
•	Передача утилиту информации об изменении статуса агента
•	Передача утилиту данных, полученных от агента
•	Получение от утилиты команды на запрос данных с агентов

\item[ПО «Монитор» (сервис): блок работы с БД]

Записывает в БД информацию о новых агентах и изменении статуса агентов и при необходимости считывает эту информацию.
\item[ПО «Монитор» (сервис): рабочий блок]

Блок, обеспечивающий работу приложения, и использующий функции из остальных блоков.
\item[ПО «Монитор» (утилита): GUI]

Графический интерфейс приложения. Отображает на экране данные и позволяет пользователю взаимодействовать с программой.
\item[ПО «Монитор» (утилита): настройки]

Изменение настроек, хранящихся на жёстком диске в *.ini файлах. Настройки включают в себя время автообновления данных, список рабочих станций, с которых нужно собирать данные, и типы данных, которые нужно отображать.
\item[ПО «Монитор» (утилита): блок связи с сервисом]

Выполняет следующие задачи:
•	Получение от сервиса информации об изменении статуса агента
•	Получение от сервиса данных, полученных от агента
•	Передача в сервис команды на запрос данных с агентов

\item[ПО «Монитор» (сервис): блок работы с БД]

Считывает из БД информацию о новых агентах и изменении статуса агентов.
\item[ПО «Монитор» (сервис): рабочий блок]

Блок, обеспечивающий работу приложения, и использующий функции из остальных блоков.

\end{description}