\section{Функциональность комплекса }

\subsection{Функциональность агента}

Агент предназначен для сбора информации с рабочей станции о ей работе и отправки этой информации на монитор по получении соответствующего запроса. Агент должен иметь возможность работать только с 1 монитором, определяемым заданным IP адресом или FQDN в конфигурационном файле агента.

При запуске сервис агента автоматически связывается с монитором посредством отправки сообщения по сети и тем самым оповещает монитор о своей готовности к работе. При этом сервис агента стартует автоматически как после установки ПО  «Агент», так и после включения рабочей станции (ПО  «Агент» должно быть автоматически установлено).

В данном разделе подробно описаны функциональные возможности ПО  «Агент».

\subsubsection{FR1: Получение данных агентом}

ПО  «Агент» должен получать данные с рабочей станции, на которой он установлен:
\begin{itemize}
\item список запущенных процессов с указанием:
	\begin{itemize}
	\item имя пользователи, запустившего процесс
	\item затраты CPU данным процессом (относительное значение в \% от общего доступного CPU)
	\item затраты оперативной памяти данным процессом (абсолютное значение используемой в данный момент памяти, так называемый рабочий набор,  в Кб и относительное значение в \% от общей доступной памяти рабочей станции)
	\end{itemize}
\item общее значение используемой в данный момент оперативной памяти (абсолютное значение в Кб и относительное значение в \% от общей доступной памяти рабочей станции)
\item общее значение используемого CPU (относительное значение в \% от общего доступного CPU)
\end{itemize}
Получение данных должно происходить по запросу, полученному агентом от монитора

\subsubsection{FR2: Автоматический старт сервиса агента после установки}

Сервис агента должен автоматически стартовать сразу после установки ПО «Агент» на рабочую станцию, то есть быть готовым к взаимодействию с монитором путйм получения и отправки сообщений и запросов, а также к получению данных (FR1).

\subsubsection{FR3: Автоматический старт сервиса агента после запуска рабочей станции}

Сервис агента должен автоматически стартовать сразу после включения рабочей станции, то есть быть готовым к взаимодействию с монитором путём получения и отправки сообщений и запросов, а также к получению данных (FR1).

\subsubsection{FR4: Задание монитора}

При установке агента пользователь должен иметь возможность задать монитор, с которым будет взаимодействовать агент (необходимо задание IP адреса или FQDN). Данные о мониторе должны быть записаны в конфигурационный сайл агента.

\subsubsection{FR5: Изменение монитора}

Пользователь должен иметь возможность изменить монитор, с которым должен взаимодействовать агент, путём изменения конфигурационного файла. Конфигурационный файл не должен изменяться пользователем «вручную», с помощью сторонних программ, то есть в состав ПО «Агент» входит утилита, позволяющая пользователю работать с конфигурационным файлом.

\subsubsection{FR6: Сообщение монитору о готовности}

Сервис агента должен автоматически отправлять сообщение монитору о его готовности к работе. Сообщение должно отправляться каждый раз после запуска сервиса агента. Сообщение отправляется монитору, указанному в конфигурационном файле агента.

Также агент должен получить сообщение от монитора, подтверждающее работу с агентом. В случае, если такое сообщение не было получено в течении Т секунд, сообщение о готовности к работе должно быть отправлено повторно.

По умолчанию Т=10.

\subsubsection{FR7: Изменение интервала времени}

Пользователь должен иметь возможность изменить значение интервала времени Т, в течении которого агент ждёт ответ от монитора на отправленное им сообщение о готовности к работе, путём изменения конфигурационного файла. Значение Т указывается в секундах. Конфигурационный файл не должен изменяться пользователем «вручную», с помощью сторонних программ, то есть в состав ПО «Агент» входит утилита, позволяющая пользователю работать с конфигурационным файлом.

\subsubsection{FR8: Отправка информации}

По запросу монитора агент должен собрать информацию (FR1), после чего отправить их на монитор. 

При этом монитор, с которого поступил запрос, должен совпадать с монитором, заданным в конфигурационном файле (сравнение идёт по IP адресу или FQDN).

Запросы на получение информации от мониторов, отличных от указанного в конфигурационном файле, должны игнорироваться.

\subsubsection{FR9: Ведение журнала агента}

Агент должен вести журнал событий, в котором должны быть записаны с меткой времени сообщения о следующих событиях:
\begin{itemize}
	\item установка агента
	\item запуск сервиса агента
	\item изменение конфигурационного файла (за исключением случаев, когда это было сделано с помощью сторонних программ)
	\item запрос информации от монитора, не совпадающего с монитором, указанном в конфигурационном файле
	\item штатное завершение работы агента
	\item аварийное завершение работы агента
\end{itemize}

На рабочей станции существует выделенная папка, в котором хранятися журналы агента. Каждый журнал представляет собой файл, формат которого описан в ... . Путь к файлу, в который в данный момент происходит запись событий, хранится в конфигурационном файле. Запись в новый файл журнала начинается после отправки прежнего файла на монитор (FR10).

\subsubsection{FR10: Отправка журнала}

По запросу монитора агент должен отправить журнал на монитор. 

При этом монитор, с которого поступил запрос, должен совпадать с монитором, заданным в конфигурационном файле (сравнение идёт по IP адресу или FQDN).

Запросы на получение журнала от мониторов, отличных от указанного в конфигурационном файле, должны игнорироваться.

После отправки текущего журнала запись событий начинается в новый файл, путь к которому прописывается в конфигурационном файле взамен старого.

\subsubsection{FR11: Оповещение о выключении}

Если сервис агента завершает свою работу штатным образом (отключение сервиса или выключение рабочей станции), на монитор должен быть отправлено сообщение о завершении работы агента.

\subsection{Функциональность монитора}

Монитор предназначен для централизованного сбора информации о рабочих станциях с агентов и отображения этой информации. Монитор должен иметь возможность работы со множеством (до N) агентов.
%число N нужно обсудить
Агенты, с которыми работает (то есть у которых запрашивает информацию и получает её) монитор, определяются через получение от них сообщения о готовности. Монитор также отслеживает текущее состояние агентов.

Сервис монитора стартует автоматически после как после установки ПО «Монитор» на центральную станцию, так и после её включения ( ПО «Монитор» должно быть предвательно установлено), но при этом запрашивает информацию о рабочих станциях только при включённой графической утилиты.

Графическая утилита отображает состояшие агентов, а также полученную от них информацию о рабочих станциях.  При отображении данных о рабочих станциях монитор использует фильтры, заданные пользователем. 

В данном разделе подробно описаны функциональные возможности ПО  «Монитор».

\subsubsection{FR12:}

%\item[ПО «Монитор» (сервис): блок общения с ПО «Агент»]

Выполняет следующие задачи:
•	Передача в «ПО Агент» запросов на получение данных
•	Передача в «ПО Агент» запросов на установление соединения
•	Получение от «ПО Агент» собранных данных
•	Приём полученных от ПО «Агент» сообщений о его включении/выключении 
•	Отправка ответов на сообщение о включении
•	Обработка мультипоточности запросов и формирование из них очереди

%\item[ПО «Монитор» (сервис): блок общения с графической утилитой]

Выполняет следующие задачи:
•	Передача утилиту информации об изменении статуса агента
•	Передача утилиту данных, полученных от агента
•	Получение от утилиты команды на запрос данных с агентов

%\item[ПО «Монитор» (сервис): блок работы с БД]

Записывает в БД информацию о новых агентах и изменении статуса агентов и при необходимости считывает эту информацию.
%\item[ПО «Монитор» (сервис): рабочий блок]

Блок, обеспечивающий работу приложения, и использующий функции из остальных блоков.
%\item[ПО «Монитор» (утилита): GUI]

Графический интерфейс приложения. Отображает на экране данные и позволяет пользователю взаимодействовать с программой.
%\item[ПО «Монитор» (утилита): настройки]

Изменение настроек, хранящихся на жёстком диске в *.ini файлах. Настройки включают в себя время автообновления данных, список рабочих станций, с которых нужно собирать данные, и типы данных, которые нужно отображать.
%\item[ПО «Монитор» (утилита): блок связи с сервисом]

Выполняет следующие задачи:
•	Получение от сервиса информации об изменении статуса агента
•	Получение от сервиса данных, полученных от агента
•	Передача в сервис команды на запрос данных с агентов

%\item[ПО «Монитор» (сервис): блок работы с БД]

Считывает из БД информацию о новых агентах и изменении статуса агентов.
%\item[ПО «Монитор» (сервис): рабочий блок]

Блок, обеспечивающий работу приложения, и использующий функции из остальных блоков.

%\end{description}