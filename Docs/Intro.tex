\section{Введение}

\subsection{Цель}
В данном документе подробно описываются все внешние проявления и сценарии поведения разрабатываемого в рамках проекта «Homer J.»  программного обеспечения (далее «комплекс», ПО, ПО «Homer J.»), его приложений, частей и подсистем. Наряду с этим приводится перечень нефункциональных требований, проектных ограничений и других аспектов, необходимых для полного и всестороннего описания всех требований участников к проектному решению.

\subsection{Область действия}
Документ разработан в рамках проекта «Homer J.» и предназначен для использования участниками данного проекта.

\subsection{Определения и сокращения}

ПО --- программное обеспечение

ПК --- персональный компьютер

ОС --- операционная система

Центральная станция --- ПК, с которого осуществляется централизованный сбор информации о работе подконтрольных объектов).

Подконтрольный объект (ПКО, рабочая станция) --- персональный компьютер, данные о работе которого собирает ПО «Homer J.».


\subsection{Краткое описание}
Данный документ содержит следующие разделы:

\begin{itemize}
\item Общее описание

Содержит общее описание проекта, а также допущения и условия, которые, так или иначе, воздействуют на техническую реализацию и использование проектного решения

\item Спецификафия требований

Содержит детальное описание всех требований (функциональных и нефункциональных) к комплексу.
\end{itemize}

