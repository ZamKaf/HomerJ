\section{Общее описание}

Основные задачи системы:
\begin{itemize}
\item	Сбор данных об использовании процессами ресурсов рабочих станций
\item	Передача этих данных с нескольких рабочих станций на центральную машину
\item	Отображение данных в удобном для пользователя виде
\end{itemize}

\subsection{Элементы комплекса}
\begin{itemize}
\item	ПО «Агент» (агент)
\item	ПО «Монитор» (монитор)
\end{itemize}

ПО «Агент» —  это сервис, запущенный на удалённых рабочих станциях. Он собирает необходимые данные и передаёт их по сети в ПО «Монитор».

ПО «Монитор» — это приложение, которое находится н центральной рабочей станции. Она собирает данные от агентов и отображает эти данные на экране. ПО «Монитор» состоит из двух частей: сервиса и графическая утилита. Сервис отвечает за общение с агентами и сбор данных, утилита позволяет пользователю редактировать настройки и просматривать собранную информация.

\subsection{Список пользователей системы}
Оператор монитора — человек, который следит за отображаемыми в ПО «Монитор» данными.

\subsection{Возможности пользователей}
Единственный пользователь системы может запросить данные с удалённых машин или установить периодичность автоматического обновления этих данных. Так же он может редактировать список ПКО, с которых нужно сейчас собирать данные, и список данных, которые нужно отображать.

\textsl{Раздел будет доработан}