\section{Общее описание}

ПО «Homer J.» предназначено для централизованного сбора информации об исполняемых процессах и используемых ресурсах с рабочих станций. Информация собирается автомачески через заданные интервалы времени.

Основные задачи ПО:
\begin{itemize}
\item	Сбор данных о запущенных на рабочих станциях процессах, об использовании ресурсов (CPU, память) рабочей станции каждым из процессов, а также общего использования ресурсов. 
\item	Передача собранных данных на центральную станцию
\item	Отображение собранных данных на центральной станции в удобном для пользователя виде
\end{itemize}

\subsection{Элементы комплекса}
\begin{itemize}
\item	ПО «Агент»
\item	ПО «Монитор» 
\end{itemize}

ПО «Агент» (далее агент) --- устанавливаемое на рабочую станцию программное обеспечение, предназначенное для сбора информации с рабочей станции, а также её передачи по сети в ПО «Монитор». Состоит из сервиса (далее «Агент»-сервис), автоматически запускаемого при старте рабочей станции, а также конфигурационного файла. 

ПО «Монитор» (далее монитор) --- устанавливаемое на центральную станцию программное обеспечение, предназначенное для сбора информации с агентов, отображения этих данных, настройки работы комплекса и ведения журналы его работы. Основными функциональными частями ПО «Монитор» являются сервис (далее «Монитор»-сервис) и графическая утилита (далее «Монитор»-GUI или утилита). «Монитор»-сервис отвечает за общение с агентами: регистрацию, запрос и сбор данных. Утилита позволяет пользователю редактировать настройки и просматривать собранную информация.

Подробное описание состава комплекса приведено в ...

\subsection{Список пользователей комплекса}
Пользователь комплекса --- любой человек, обладающий достаточными знаниями и полномочиями для установки агентов на рабочии станции, установки монитора на центральную станцию, настройки комплекса, а также осуществляющий просмотр отображённых в ПО «Монитор» данных.Также пользователь комплекса  может запросить данные с рабочих станций, установить периодичность автоматического обновления этих данных. Так же он может редактировать список рабочих станций, с которых нужно сейчас собирать данные, и список данных, которые нужно отображать.

ПО «Homer J.» не предполагает разделение ролей пользователей, а также контроль доступа пользователей к его элементам (см. ограничение ...)

